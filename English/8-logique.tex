% éèàùôê

\chapter{Histoire de menteurs}

%%%%%%%%%%%%%%%%%%%%%%%%%%%%%%%%%%%%%%%%%%%%%%%%%%%%%%%%%%%%%%%%%%%%%%%%%%%%%%%%
%                                    TP 4                                      %
%%%%%%%%%%%%%%%%%%%%%%%%%%%%%%%%%%%%%%%%%%%%%%%%%%%%%%%%%%%%%%%%%%%%%%%%%%%%%%%%


\setlength{\parskip}{2ex}
\setlength{\parindent}{0pt}

\section{Le puzzle}
\label{sec:mensonge}

Parent1 et Parent2 forment un couple hétérosexuel, mais on ne sait pas qui est la femme ni
qui est l'homme ! Ces deux personnes ont un enfant Enfant dont on ne connaît pas le
sexe.

Le seul critère permettant de discerner femmes et hommes est le suivant :

\begin{enumerate}
\item Les femmes disent toujours la vérité.
\item Les hommes alternent systématiquement vérité et mensonge.
\end{enumerate}

(Selon vos convictions, vous pouvez bien sûr adapter l'énoncé \ldots)

On demande à Enfant s'il est un homme ou une femme :

\begin{tabbing}
  \hspace{2.5em} Enfant affirme : Arrheu, arrheu~!
\end{tabbing}
On se tourne alors vers ses parents Parent1 et Parent2.
\begin{tabbing}
  \hspace{2.5em}\= Parent1 affirme : Enfant vous dit qu'elle est une femme.\\
                \> Parent2 affirme : Enfant est un homme puis \dots\\
                \> Parent2 affirme : Enfant ment.
\end{tabbing}

On cherche à savoir qui est le père, qui est la mère et quel est le sexe de l'enfant.


\section{Modélisation}

Modéliser ce puzzle logique en utilisant les contraintes logiques de la
bibliothèque \emph{ic}. Pour cela, identifier les variables et leur domaine.

\begin{question}
« Les femmes disent toujours la vérité. ». \\
Définir le prédicat \code{affirme/2} tel que \code{affirme(?S,?A)} pose
la contrainte : l'affirmation \code{A} est vraie si \code{S} est une
femme.
\end{question}

\begin{question}
« Les hommes alternent systématiquement vérité et mensonge. » \\
Définir le prédicat \code{affirme/3} tel que \code{affirme(?S,?A1,?A2)}
pose la contrainte~: si \code{S} est un homme, les affirmations $A_1$
et $A_2$ sont l'une vraie, l'autre fausse.
\end{question}
\begin{tabbing}
  \hspace{2.5em}\= AffE  est l'affirmation de l'enfant.\\
                \> AffEselonP1 est l'affirmation de l'enfant selon Parent1.\\
                \> AffP1 est l'affirmation de Parent1.\\
                \> Aff1P2 est la première affirmation de Parent2.\\ 
                \> Aff2P2 est la seconde affirmation de Parent2.
\end{tabbing}

Chacune de ces affirmations appartient au domaine booléen $\{0,1\}$ (représenté par l'intervalle entier [0..1]) :
\begin{tabbing}
  \hspace{2.5em}\= AffEselonP1 : \hspace{1em}\= Enfant est une femme\\
                \> AffP1       :             \> AffEselonP1 = AffE\\
                \> Aff1P2      :             \> Enfant est un homme\\
                \> Aff2P2      :             \> AffE = 0
\end{tabbing}

\begin{question}
  Définir le domaine symbolique des variables \code{Parent1}, \code{Parent2} et \code{Enfant}, puis écrire le prédicat qui contraint
  le domaine de l'ensemble des variables.  
\end{question}

\begin{question} 
  Poser les contraintes sur les variables et définir un prédicat de labeling pour les valeurs symboliques (comme pour le TP \ref{tp2}). Utilisez ce prédicat pour résoudre le problème.
\end{question}

% \section{Compte-rendu}

% \subsection{À rendre}

% \begin{enumerate}
% \item La réponse au problème posé.
% \item Le code \eclipse{} \emph{commenté}. Donnez les requêtes ainsi que les réponses du système. 
% \end{enumerate}


