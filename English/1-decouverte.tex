

\chapter{Discovering the library for constraint programing.}

%\section{Prise en main de \emph{ic}}
%\label{sec:domaines-finis}



\section{From \prolog{} to \prolog{}+ic}

\subsection{Constraints over trees}

A rich and refined vehicle buyer wishes to purchase a car and a boat of the same color.
The car he has chosen can be find in red, light green, gray or white (single color will be represented by an atom, for example `` red '' and an existing color in various shades by a term such as `` green (light) ').
The boat model that interests may be available in green, black or white.

\begin{question}
 Write in \prolog{} pure, the sentence \\\code{choixCouleur(?CouleurBateau, ?CouleurVoiture)} which is true iff the colors chosen for the boat and the car are the same and are part of existing choices.
\end{question}

\begin{question}\label{QC1}
Explain why \prolog{} can be considered as a constraint solver on the domain tree.
\end{question}

\subsection{\prolog{} do not manage Maths (but it's an awesome calculator)}
To produce its flagship product, a company needs to capacitors and resistors. Electrical equipment provider can provide between 5,000 and 10,000 resistors and capacitors between 9000 and 20000.
For the next order, the company plans to order more than resistors capacitors.

\begin{question}
 Write the sentence \code{isBetween(?Var,+Min,+Max)} which sets a value for  \code{Var} and it's true iff \code{Var} has one value in between \code{Min} and \code{Max}.
\end{question}
\begin{question} 
Use the sentence \code{isBetween} and \code{>=} to define \\\code{commande(-NbResistance, -NbCondensateur)} which sets the number of resistors and capacitors in order to meet the problem statement.
\end{question}
\begin{question}\label{QCA1}
  rely on \eclipse{}'s debugger to design the \prolog{} search tree when searching the solution \code{commande(-NbResistance, -NbCondensateur)}.  \\
 (NB: we do not ask for the trace in the report,
   only the tree.)
\end{question}
\begin{question}\label{QC3}
Justify the title of the year (tracks: what will happen if we put the predicate~\code{>=} before the calls to \code{isBetween} ? Why we use and call it test-driven development?)
\end{question}

\subsection{The solver \code{ic} to the rescue}
\begin{question}\label{QC4}
 Change the sentence \code{isBetween} by the constraint \code{ic} \code{Var \#:: Min..Max} and \code{>=} by \code{\#>=}. See what happens in \eclipse{}. Why is this happening? 
\end{question}

\begin{question}\label{QCA2}
Use the \code{labeling} predicate to find solutions to
   problem (see section labeling page \pageref{intro:labeling})  and
   draw the new search tree \prolog{}, keep on using the tracer.
\end{question}



\section{Zoologie}
\label{sec:zoologie}

Consider \code{Chats}s cats and \code{Pies} pies. These \code{Chats} Cats cats and \code{Pies} pies total \code{Pattes}
legs and \code{Tetes} heads. The problem that concerns us therefore includes four
variables that are "linked" with numerical constraints.

These variables belong to a finite domain: a sub-set of integers
natural. Indeed the negative head numbers or $2/3$ means anything for us!

Define a sentence \code{chapie/4} where \code{chapie(-Chats,-Pies,-Pattes,-Tetes)} establishes
constraint linking the four variables.

Use this predicate to answer the following questions:
\begin{question}
How many pies and legs does it take to total five heads and two cats?
\end{question}

\begin{question}
How much to cats and pies for three times as many legs as
heads?
\end{question}

\section{Le ``OU'' en contraintes}
\label{sec:suite-periodique}
When programming with \code{\prolog{} + ic}, the `` or '' logic can be implemented in two ways:

\begin{itemize}
\item With a choice point \prolog{}.
\item With the disjunction operator \verb|or| from \emph{ic}.
\end{itemize}


\begin{question}
En utilisant successivement ces deux méthodes, définissez   le   prédicat  \code{vabs/2},   where
\code{vabs(?Val,?AbsVal)} impose the constraint  : \code{AbsVal} is the absolute value of an integer \code{AbsVal}.
Test different versions of this predicate in varying arguments.\end{question}

\begin{question}\label{QC5}
Run the query \code{X \#:: -10..10, vabs(X,Y)} with both versions of \code{vabs}. Compare and discuss the results.
\end{question}


Following is defined by:
\begin{displaymath}
    X_{i+2} = |X_{i+1}| - X_i
\end{displaymath}

\begin{question}

Define the predicate \code{faitListe(?ListVar, ?Taille, +Min, +Max)} forcing ListVar to be a size of Size list whose elements are variable \code{ic} so the domain is \code{Min..Max} .
\end{question}


\begin{question}
Define the prédicate \code{suite(?ListVar)} which takes a list and constraint the elements of that list is such a way that they are consecutive.
\end{question}

\begin{question} 
Ask a query to verify that this sequence is periodic of period 9.
\end{question}


\section{Compte-rendu de TP}

\subsection{À rendre}

\begin{enumerate}
\item Source code \eclipse{} and queries \eclipse{}
  used and the system responses.
\item The written answers questions \ref{QC1}, \ref{QC3},
  \ref{QC4} et \ref{QC5}, and the search tree for \ref{QCA1} et \ref{QCA2}.
\end{enumerate}



