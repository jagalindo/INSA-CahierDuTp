\addcontentsline{toc}{chapter}{Déroulement des travaux pratiques}
\chapter*{Déroulement des travaux pratiques}

\section*{Compte-rendus}
Les travaux pratiques se feront par binômes et chaque TP devra faire
l'objet d'un compte-rendu qui sera noté.  {\bf Ces compte-rendus
  devront être rendus au plus tard lors du TP suivant, en version
  papier et déposés sur Moodle.}
% pour 2013 il faudra donner les indications précises.

Les compte-rendus  inclueront au minimum~:
\begin{enumerate}
\item Le code final de votre solution. Remarque : les notes prendront
  en compte la qualité du code produit et des commentaires.
 \item {\bf Les tests} vous ayant permis de
   valider votre code. Chaque test sera décrit par~:
\begin{itemize}
\item Les données utilisées (si ce ne sont pas celles du texte du TP)
\item Le but executé
\item Les premières réponses d'\eclipse
\item Le temps d'exécution de la requête
\item Une indication du nombre de réponses
\end{itemize}

\item Une réponse succincte mais \textbf{rédigée} à toutes les
  questions de compréhension. N'hésitez pas à utiliser des schémas ou
  des exemples
  lorsque cela est pertinent.\\

Les deux derniers points devront être inclus dans un grand
commentaire. Le fichier déposé sur Moodle doit pouvoir être exécuté
directement par le correcteur.
\end{enumerate}
 
\section*{Structure de vos programmes}
Le code de vos TP devra respecter la structure suivante~:
\begin{itemize}
 \item prédicat principal chargé de résoudre le problème posé (ce prédicat sera enrichi tout au long du TP).
 \item prédicats chargés de poser les données
 \item prédicats de définition des variables et de leurs domaines
 \item prédicats posant les contraintes 
 \item prédicats utilitaires
\end{itemize}

\section*{Qualité du code et conventions de codage}
Vous devrez porter attention à la qualité du code, et plus particulièrement~:
\begin{itemize}
 \item Utiliser des noms de prédicats et de variables significatifs (pas de p(X,Y), \ldots)
 \item \textbf{Indenter le code} 
 \item Ne lister \textbf{qu'un seul prédicat par ligne,} même, et surtout, si
   le prédicat est court.  Ça peut sembler un gâchis de place, mais ça
   vous fera beaucoup de temps lors de la mise au point. Ça permettra
   aussi aux enseignants de vous aider plus facilement.
 \item Ne pas entrelarder son code de commentaires~: \textbf{regrouper tous les
   commentaires en entête d'un prédicat.} Un commentaire à
   l'intérieur d'un prédicat est souvent le signe qu'il faudrait
   découper en prédicats auxilliaire dont le nom pourrait véhiculer
   l'essentiel de ce qu'on voulait mettre dans le commentaire.
 \item Ne pas mettre de constantes en dur dans le programme
 \item Faire du code modulaire (pas de prédicats fourre-tout)
 \item Faire du code réutilisable lorsque cela est pertinent
\end{itemize}


\section*{Méthodologie}
Le débogage de programmes à contraintes n'est pas évident. Aussi, il
est très important de \textbf{tester vos programmes au fur et à
  mesure}.  Le processus de résolution d'un problème devra donc être
incrémental :
\begin{enumerate}
 \item Définition des données pour la mise au point (éventuellement
   plus restreintes que les données du problème qu'on cherche
   réellement à résoudre)
 \item Définition des variables et de leur domaine
 \item Obtention d'une solution et vérification de la cohérence de
   cette solution avec les contraintes déjà posées
 \item Ajout d'une contrainte 
 \item Obtention d'une solution et vérification
 \item Itérer sur les deux points précédents tant qu'il y a des
   contraintes à poser
 \item \label{une:sol} Obtention d'une solution prenant en compte toutes les
   contraintes
\item Itérer avec les données de l'énoncé, si les données initiales
  avaient été restreintes.\\
\end{enumerate}

Il faudra systématiquement commencer avec des données de test les plus
petites possibles et ne passer aux valeurs conséquentes qu'une fois
que le point~\ref{une:sol} est atteint.

\section*{Documentation}
La documentation d'\eclipse{} est disponible en local
\url{/usr/local/stow/eclipse6.0\_136/doc/bips}. C'est cette
documentation qu'il faut utiliser en priorité. Pour le cas où des
informations viendraient à manquer, vous pouvez aussi consulter la
documentation en ligne à l'url~: \url{http://eclipseclp.org/doc/bips}.\\

{\bf Ajoutez des signets lors du premier TP sur ces pages et ayez
systématiquement le manuel de référence ouvert lors des TP (cf annexe
du cours).}


\section*{Mise au point}

Pour finir, voici quelques points qui peuvent vous aider à trouver des
erreurs dans vos programmes~:

\begin{itemize}
\item {\bf Faîtes systèmatiquement disparaître les warnings du
    compilateur}. En effet, la plupart du temps les avertissements
  remontés par le compilateur sont des symptômes d'erreurs.\\  

%   De plus, il ne faut pas prendre l'habitude d'avoir de nombreux
%   avertissements (ex : variables ``singleton'') car cela vous
%   empêche de voir lorsque le compilateur remonte un problème grave.\\

\item {\bf Méfiez vous des ``delayed goals''}. Il est tout à fait
  normal d'avoir des buts suspendus tant que vous ne faîtes que poser
  les contraintes.  En revanche, lorsqu'\eclipse{} dit avoir trouvé
  une solution à ces contraintes, s'il reste des buts en attente, il
  est probable que toutes les contraintes n'aient pas été prises en
  compte. On ne peut donc pas faire confiance au résultat, tout
  particulièrement s'il y a eu du ``Branch and Bound''.\\

\item {\bf Utilisez le traceur}. Cela permet de voir exactement ce qui
  se passe à l'exécution.

  L'annexe du cours contient des indications concrètes pour utiliser
  le traceur d'ECLiPSe. Cela peut s'avérer très utile pour comprendre,
  par exemple, les erreurs d'exécution ou les itérateurs mal
  imbriqués.  Il est crucial de tracer des exécutions où les données
  de test ont été réduites le plus possible pour réduire la taille de
  la trace.

\end{itemize}
