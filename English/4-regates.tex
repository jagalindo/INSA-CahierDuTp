% éèàùôê

\chapter{The races}


\section{Problem Statement}
\label{sec:enonce-probleme}


The organizers of a regatta can have 3 yachts each to host a number of teammates. The  accommodation capacities for the yachts are:
\begin{center}
    \begin{tabular}{|c|c|c|}
        \hline
	7&6&5\\
        \hline
    \end{tabular}
\end{center}



4 teams responded to the invitation of the organizers. The size of these
invited teams is given by the following table:

\begin{center}
    \begin{tabular}{|c|c|c|c|}
        \hline
	5&5&2&1\\
        \hline
    \end{tabular}
\end{center}


The puzzle of the organizers then is to try to make a schedule
with the following requirements:

\begin{itemize}

\item The regatta takes place in 3 successive confrontations in which
       3 sailboats compete;

\item invited teams are spread over the 3 boats so
       that :

    \begin{enumerate}

    \item members of the same team stay together,
    \item sailboats of accommodation capacities are met,
    \item during successive confrontations no team returns two time on the same boat,
    \item during successive confrontations no team finds
           with the same partner team.
    \end{enumerate}

\end{itemize}

\clearpage

The goal is to establish a schedule two entries for any confrontation and
for any guest team indicates the number of host sailboat. the schedule
below is a possible answer. It states that during the confrontation 3
Team No. 4 is on the boat 2; it is then Team Partner
2.

\begin{center}
    \begin{tabular}{|c|c|c|c|c|c|c|c|}
        \hline
        \backslashbox{Équipe}{Confrontation} &
         1 &  2 &  3 \\ \hline
         1 &  1 &  2 &  3  \\ \hline
         2 &  3 &  1 &  2  \\ \hline
         3 &  2 &  3 &  1  \\ \hline
         4 &  1 &  3 &  2  \\ \hline
       \end{tabular}
\end{center}


\section{Looking for a solution using ECLiPSe}
\label{sec:vers-une-solutionRegates}

\subsection{A first partial solution}
The problem data are in the form of two vectors: the
vector of terminal facilities and the vector of the sizes of invited teams.
The variables of the problem are the number of \code{NbEquipes * NbConfrontations}. These variables,
representing a host sailboat, have a field \verb|1..NbBateaux|. In the following, the variables are contained in an array with \code{NbEquipes} rows and \code{NbConfrontations} columns.

\begin{question}
 Define the predicate \\\code{getData(?TailleEquipes, ?NbEquipes, ?CapaBateaux, ?NbBateaux, ?NbConf)} \\ that unifies the variables as parameters with the data of problems.
\end{question}


\begin{question}
Define the predicate \code{defineVars(?T, +NbEquipes, +NbConf, +NbBateaux)} which unifies \code{T} the array of variables described above and forced the field variables.
\end{question}

\begin{question}
  Define the predicate \code{getVarList(+T,?L)} which constructs the
  list \code{L} with the variables in the table\code{T}. The variable list must contain the variables of the first column followed by those in the second column, etc. (a test that shows that the order is correct).
  \end{question}

\begin{question}
 Define the predicate  \code{solve(?T)} qui résoud le problème des régates où seules les contraintes de domaines sont posées. 
Vérifiez que les réponses rendues par eclipse vous semblent correctes. 
\end{question}

\subsection{Taking into account problem constraints}

\begin{question}
 Define the predicate\code{pasMemeBateaux(+T, +NbEquipes, +NbConf)}which requires that the same team will not return twice on the same boat.
Modify the predicate \code{solve} to take into account this new constraint and check that the answers given by eclipse are consistent.\end{question}

\begin{question}
 Define the predicate\code{pasMemePartenaires(+T, +NbEquipes, +NbConf)} which requires that the same team is not found twice with the same team.
Modify the predicate \code{solve} to take into account this new constraint and check that the answers given by eclipse are consistent.
\end{question}

\begin{question}
 Define the predicate \\ \code{capaBateaux(+T, +TailleEquipes, +NbEquipes, +CapaBateaux, +NbBateaux, +NbConf)} which verifies that the capacity of ships are met at every confrontation.
Modify the predicate \code{solve} to take into account this new constraint and check that the answers given by eclipse are consistent.\end{question}

\section{let's go ahead with a real-world size problem}

The data used so far are only test data; it would be quite possible to solve the problem without using a computer.
In this section, we consider data more representative of the puzzle that can be organizing a regatta:

We have 13 sailboats whose capacities are listed below:
\begin{center}
    \begin{tabular}{|c|c|c|c|c|c|c|c|c|c|c|c|c|}
        \hline
        10&10&9&8&8&8&8&8&8&7&6&4&4\\
        \hline
    \end{tabular}
\end{center}
  
The number of teams going to 29. The size of each team:
\begin{center}
    \begin{tabular}{|c|c|c|c|c|c|c|c|c|c|c|c|c|c|c|}
        \hline
        7 & 6 & 5 & 5 & 5 & 4 & 4 & 4 & 4 & 4 & 4 & 4 & 4 & 3 & 3\\
        \hline
        \multicolumn{15}{c}{}\\
        \cline{1-14}
        2 & 2 & 2 & 2 & 2 & 2 & 2 & 2 & 2 & 2 & 2 & 2 & 2 & 2 & \multicolumn{1}{c}{}\\
        \cline{1-14}
    \end{tabular}
\end{center}

The regatta takes place in 7 confrontations. First try with
less confrontation and gradually try to 7
confrontations.


Finding a solution to this new problem can be very long! he
you should program your own labeling. Idea to dig:
when you have the list of 29 variables of a confrontation without
doubt it is preferable to get faster a first solution,
reorder the list so as to alternate small and large team.

\begin{question}
 Implement \code{getVarListAlt(+T,?L)} to solve this problem in a reasonable time.
 \end{question}

The following configuration is a solution of the problem:
\begin{center}
    \begin{tabular}{|c|c|c|c|c|c|c|c|}
        \hline
        \backslashbox{Équipe}{Confrontation} &
         1 &  2 &  3 &  4 &  5 &  6 &  7 \\ \hline
         1 &  1 &  2 &  3 &  4 &  5 &  6 &  7 \\ \hline
         2 &  2 &  1 &  4 &  3 &  6 &  5 &  8 \\ \hline
         3 &  3 &  4 &  1 &  2 &  8 &  7 &  9 \\ \hline
         4 &  4 &  3 &  5 &  1 &  2 &  9 & 10 \\ \hline
         5 &  5 &  6 &  2 &  7 &  1 &  3 &  4 \\ \hline
         6 &  6 &  5 &  7 &  2 &  1 &  4 &  3 \\ \hline
         7 &  6 &  7 &  8 &  5 &  2 &  1 & 11 \\ \hline
         8 &  7 &  5 &  6 &  8 &  9 &  1 &  2 \\ \hline
         9 &  8 &  9 &  7 & 10 &  4 & 11 &  5 \\ \hline
        10 &  8 & 10 &  9 &  6 & 11 &  3 &  2 \\ \hline
        11 &  9 &  8 & 12 & 13 &  7 &  2 &  6 \\ \hline
        12 & 10 &  9 &  8 & 11 & 13 & 12 &  1 \\ \hline
        13 & 12 & 13 & 11 &  9 & 10 &  8 &  1 \\ \hline
        14 & 11 & 10 & 13 &  3 &  8 &  9 &  4 \\ \hline
        15 & 11 & 12 & 10 &  7 &  3 & 13 &  9 \\ \hline
        16 & 13 & 11 & 10 &  9 &  6 &  1 &  5 \\ \hline
        17 & 13 &  8 & 11 & 12 &  4 & 10 &  3 \\ \hline
        18 & 10 & 11 &  9 &  8 & 12 &  2 &  3 \\ \hline
        19 &  9 & 11 &  6 & 12 & 10 &  5 & 13 \\ \hline
        20 &  9 &  7 & 10 & 11 & 12 &  8 &  2 \\ \hline
        21 &  7 &  8 &  9 & 10 &  3 &  4 &  1 \\ \hline
        22 &  7 &  6 &  5 &  9 & 11 &  2 &  8 \\ \hline
        23 &  5 &  7 &  1 &  8 &  3 & 10 & 13 \\ \hline
        24 &  4 &  1 &  6 &  5 &  3 &  2 & 12 \\ \hline
        25 &  3 &  2 &  4 &  1 &  7 & 11 & 12 \\ \hline
        26 &  3 &  1 &  2 &  6 &  9 & 10 & 11 \\ \hline
        27 &  2 &  4 &  3 &  1 &  9 &  8 &  6 \\ \hline
        28 &  2 &  3 &  1 &  6 &  7 &  4 &  5 \\ \hline
        29 &  1 &  3 &  2 &  5 &  4 &  7 &  6 \\ \hline
    \end{tabular}
\end{center}
\section{homework}

\subsection{To deliver}

\begin{enumerate}
\item The answer to the problem.
\item The code \eclipse{} \emph{with comments}. Give eclipse requests
   and the system responses.
   \end{enumerate}

