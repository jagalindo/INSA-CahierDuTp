\chapter{Contraindre puis chercher}

\noindent Generally the constraints programs include
two stages, the first is to express the problem and the second
is to find a variable assignment that satisfies
constraints, ie a solution to the problem. We may want
find any solution, all solutions or solution
optimal (or `good 'according to a given criterion). In this problem we
intéresserons the latter type of solution.

%\section{Totalement immoral !}
\section{Le problème}
\label{sec:immoral}

Consider the following problem: a production unit can produce
various mobile phone products. The manufacture of the so $ S $
$ N $ mobilizes workers and produces mobile phones $ P $ a day,
for each of these phones benefit is $ E $ euros. The unit has
22 technicians.

It has the following data:

\begin{center}
    \begin{tabular}{|l|r|r|r|r|r|r|r|r|r|}
        \hline
        Sorte & $S_1$ & $S_2$ & $S_3$ & $S_4$ &
                $S_5$ & $S_6$ & $S_7$ & $S_8$ & $S_9$\\ \hline
        Nb. Tech &5&7&2&6&9&3&7&5&3\\ \hline
        Qté jour&140&130&60&95&70&85&100&30&45\\ \hline
        Bénef&4&5&8&5&6&4&7&10&11\\ \hline
    \end{tabular}
\end{center}



We want to know what you have to start manufacturing for the benefit.

\newpage
\section{Modéliser et contraindre}

Before coding it is imperative model, that is to say to
the problem equations. To this we must look for a model
general. The problem data are in the form of three
values of vectors: \code{Techniciens}, \code{Quantite},
\code{Benefice}. The variables form a vector\code{Fabriquer}  which
the values will be determined by your program. These values are
either 1 or 0, depending on whether or not the lance production.

\noindent The program must be composed as follows:

\begin{enumerate}
\item part données (faits, variables);
\item part « prédicats de services » (produit scalaire, \dots);
\item part constraints predicates (equations expressing the problem of properties);
\item part resolution constraints, a predicate that:
    \begin{itemize}
    \item « récupère » data and variables of the problem,
    \item asks the constraints on these variables.
    \end{itemize}
\end{enumerate}

\begin{question} \label{TPCPC_Qdeb} 
  Write predicates that define the three vectors
   values and the vector of variables.
\end{question}

\begin{question}
 Define the predicates which express from
   defined vectors:
\begin{itemize} 
  \item the total number of necessary workers
  \item the vector of total profit by kind of telephone
  \item the total profit
\end{itemize} 
These equations represent predicates using vector operations.
\end{question}

\begin{question}
 Define the predicate: \\
  \code{pose\_contraintes(?Fabriquer, ?NbTechniciensTotal, ?Profit)} which
   poses constraints, then call and list the solutions
   respect these constraints (there are many!).
\end{question}


\section{Optimiser}

\subsection{Branch and bound dans ECLiPSe}
To make the optimization under the constraints, the algorithm \emph{branch and bound} is used. 
\newpage
In \eclipse{}, the predicate \code{minimize/2} from the library
\emph{branch\_and\_bound} is an implementation of that algorithm. the
first parameter is a goal. This object generates a search tree which performs
the list of possible solutions. The second parameter is a variable representing the
cost of the solution. It is this variable that should be minimized. 

The predicate \code{minimize / 2} works on the following principle:
as soon as a solution is found, it is stored and research
continues with the additional constraint that the cost is less
than stored. By repeating this process until no longer find
solution, an optimal solution is obtained.

NB: \code {minimize / 2} requires that the object instantiates the variable
cost but not the rest of the variables. In this case, it may
there are no solutions to the problem to obtain the value
optimum found. The following example illustrates this property:
\\\verb|[X,Y,Z,W] #:: [0..10], X #= Z+Y+2*W , X #\= Z+Y+W|

\begin{question}\label{Qmin}
  Use \code{minimize/2} to find the smallest value
   \code{X} solution of this example. Try the labeling only
   \code{X} and finally \code{[X, Y, Z, W]}. Explain the two responses
   made by different \eclipse{}. What must always be in
   the purpose?
\end{question}

\subsection{Application à notre problème}

\begin{question}
Call the predicate poses constraints and seek
   "optimal" solution (maximizing profit) and meets the
   constraints posed.
\end{question}

Times change. The mobile market begins to saturate. shareholders
worry. So the company policy fits. We want to produce
less but earn at least 1 \, 000 per day. We want to keep making
mobile lines that ensure this goal but keeping the minimum
workers!

\begin{question} \label{TPCPC_Qfin}
Use the previous approach to address this new problem. There
not much to do (same data, same services predicates, \dots) !
\end{question}

\section{Compte-rendu}

\subsection{À rendre}

\begin{enumerate}
\item The answer to the problem.
\item The commented code. Deliver the queries \eclipse{} and
  the system responses.
\item The detailed answer to the question \ref{Qmin}.
\end{enumerate}

