% éèàùôê

\chapter{Contraintes logiques} \label{tp2}
In a general way, a constraint is expresed by an elementary operator such as X \#> Y. This is a contraint is composed of multiple primitices linked with logic connectors:~\code{and},
\code{or}, \code{=>}, \code{\#=} et \code{neg}. We also call them logic constrints.

We call symbolic constraints those refering to variables within an integer domain.In that case, the domain is defined by multiple terms such as ~:
\verb|Day &:: week|, with \verb|week| the symbolic domain represent the set of \verb|{mo, tu, we, th, fr, sa, su}|.

\section{Logic puzzle}

To solve this puzzle you need two libraries.

\begin{itemize}
 \item the library \emph{ic\_symbolic} for symbolic constraints (which operators start with
\code{\&})
\item lthe library \emph{ic} for logical connectors and constraints on integer variables (whose operators begin with \code{\#})
\end{itemize}
A predicate \code{alldifferent(+List)} exists in both
libraries and implements global constraint which forces
variables in the list \code{List} to have a different value in
the same field. The list must contain only variables
same field.

NB: When a predicate with the same name and is in the same arity
two libraries, use the syntax \verb|bibliothèque:prédicat|
to specify which one to use.


\subsection{Enoncé}

A street nearby houses contains 5 different colors. the
residents of each house are of different nationalities,
possess different animals, different cars and all
a different favorite drink. In addition, the houses are numbered
January to May in the order they are in the street.

\newpage

We know more about these houses:

\begin{tabular}{c@{ }l}
(a) & The Englishman lives in the red house \\
(b) & The Spaniard owns a dog \\
(c) & The person living in the green house drinks coffee \\
(d) & The Ukrainian drinks tea \\
(e) & The green house is located just to the right of the white house \\
(f) & BMW's driver has snakes \\
(g) & The inhabitant of the house has a yellow Toyota \\
(h) & Milk is drunk in the middle house \\
(i) & The Norwegian lives in the leftmost house \\
(j) & The driver of the Ford lives next to the person who owns a fox \\
(k) & The person driving a Toyota lives next to the house where there is a horse\\
(l) & Honda driver drinking orange juice \\
(m) & The Japanese drove a Datsun \\
(n) & The Norwegian lives next to the blue house
\end{tabular}
\paragraph{}
We want to know who owns a zebra and who drinks water.

\subsection{Modélisation}

A house is represented by a term
\\\code{m(Pays,Couleur,Boisson,Voiture,Animal,Numero)} or
\code{Pays}, \code{Couleur}, \code{Boisson}, \code{Voiture},
\code{Animal} and \code{Numero} are six variables representing the houses characteristics. The variables of the problem are
represented by a five-member list where each element is a
term \code{m(...)}. The first element of the list is the
home leftmost and is number 1, the second element
is the second house from the left and carries the
number 2, etc.

\subsection{Questions}

\begin{question} 
  Define the different symbolic domains through
   libraries from \eclipse{} (cf. documentation of \emph{ic\_symbolic}).
\end{question}

\begin{question}
 Define a predicate \code{domaines\_maison(m(...))} which forced the
   field of variables that make up a home.
\end{question}

\begin{question}
\label{numerotation}
Define the predicate \code {rue(?Street)} that unifies \code{Rue} to the
   list of houses and pose constraints
   field. NB: this predicate must set the value of variables \code{Numero} of each home.
\end{question}

\begin{question}
Write the predicate \code{ecrit\_maisons(?Rue)} defines
   iterator that retrieves each element of \code{Rue} and writes
   means of the predicate \code{writeln/1}. You will need this type
   iterator in the following questions.
\end{question}

\begin{question}
 Define the predicate \code{getVarList(?Rue, ?Liste)} allowing
   retrieve the list of variables of the problem.

 Then set a predicate labeling
  \code{labeling\_symbolic(+Liste)} using the predicate
  \code{indomain/1} de \emph{ic\_symbolic} (cf. page
  \pageref{intro:labeling}).
\end{question}

\begin{question}
 Define the predicate\code{resoudre(?Rue)} using predicates
   previous to find a solution within the constraints of
   field. Check that the solutions given by \eclipse{} are
   consistent.
\end{question}

\begin{question}
  Ask the constraints corresponding to the information of (a) to (n)
   adding to as the predicate \code{resoudre} and 
   verify that the proposed solutions are correct.
\end{question}

\begin{question}\label{Puzzle_last}
Answer the question posed in the statement ~: has a zebra and
   drinking water ~?
\end{question}

\section{Compte-rendu}

\subsection{Questions de compréhension}

\begin{enumerate}
\item In the question ~\ref{numerotation}, you set the value of variables representing the number of a house on the street. Would it have been possible not to set these values? What would have been the impact on the search for solutions?
\end{enumerate}

\subsection{À rendre}

\begin{enumerate}
\item The code \eclipse{} \emph{commented}.  Give queries
   \eclipse{} and the responses of the system and the test data when the data are not the problem.
\item the answer to the question \ref{Puzzle_last}.
\end{enumerate}
